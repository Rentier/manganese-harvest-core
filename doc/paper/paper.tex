\documentclass{article}

\usepackage[ngerman]{babel} 

\usepackage[T1]{fontenc}
\usepackage{lmodern}
\usepackage[utf8]{inputenc}
\usepackage{graphicx}

\begin{document}
% Title page

% ############################################
\section{Problembeschreibung und -analyse}

\subsection{Problembeschreibung}

Die Hauptschwierigkeit des Problems liegt daran, dass es sich eigentlich
um eine Kombination aus (mindestens) zwei Problemen handelen. Jedes Problem
für sich ist gut beschrieben und gelöst worden, doch für alles gleichzeitig
nicht.

\begin{enumerate}
\item Graph coloring - Visit stuff
\item Evacuate 
\end{enumerate}

\subsection{Annahmen über das Problem}

\begin{itemize}
\item Roboter sind punktförmig
\item Können in der Bewegung saugen
\item Keine Beschränkung, wie viele Roboter sich in einem Punkt befinden
\item Kommunikation ist instantan und ohne Berechnungszeit (Senden wie Empfangen)
\item Kommunikation zwischen Robotern kann alles sein
\item Roboter haben unbeschränkt viel Speicher
\end{itemize}

\subsection{Disclaimer}

Einschränkungen nicht simuliert, aber deren Auswirkungen

% ############################################
\clearpage
\section{Vorgehensmodell}

\begin{itemize}
\item Keep it simple
\item Start with working prototype before you do srs stuff
\item Getrennt marschieren, zusammen kämpfen (self contained subprograms, assemble at the end, bottom to top)
\item ipython notebook
\item using a lib before reinventing the wheel
\item Optimize when needed, first correct and working, then fast
\item 90 \% thinking, 10 \% coding
\item Abstract away decisions not made or which might change (geometry, polygon, agent)
\end{itemize}

% ############################################
\clearpage
\section{Problembetrachtung}

Das Problem kann unter vielen verschiedenen betrachtet 
werden. Je nachdem, in welcher Domäne der Informatik
es eingeordnet wird, gibt es unterschiedliche Lösungsansätze.

In diesem Abschnitt wird kurz beschrieben, 

\subsection{Graphenproblem}

Suche

\subsection{Optimierungsproblem}

\subsection{Maschinenlernen}

\subsection{Künstliche Intelligenz}



% ############################################
\clearpage
\section{Umgebung}

Die Modellierung des Meeresbodens hat zum Ziel, die folgenden 
Fragen zu beantworten oder einen guten Kompromiss zu finden:

\begin{enumerate}
\item Wie lassen sich eine begrenzte Anzahl an Kreisen auf einer 
unendlichen Fläche anorden, sodass die nicht von Kreisen bedeckte Fläche minimal 
ist (vergleichbar mit dem Ausstechen von Kreisen aus Keksteig)?
\item Wie lässt sich der Meeresboden unterteilen, sodass eine Simulation
möglichst einfach wird?
\end{enumerate}

Geplant ist, dass die Missionsdauer in Zeitschritt von 1s aufgeteilt wird.
In jedem Zeitschritt bewegen sich die Roboter einen geometrischen Schritt
auf dem Meeresboden weiter und säubern ihn dabei von Manganknollen. Der Vorteil
darin besteht, dass die Optimierung der Ausbeute darauf reduziert wird, 
den nächsten Schritt möglichst geeignet auszuwählen.

Dazu werden unendlichen Weiten des pazifischen Meeresgrundes
in Abschnitte in Form von Polygonen eingeteilt (parkettiert),
damit der Wertebereich der nächsten möglichen Schritte diskretisiert wird.

Die Wahl, mit welchem Polygon gearbeitet wird, hat direkte
Auswirkungen auf Genauigkeit und Leistungsfähigkeit der 
Simulation, wie im Folgenden gezeigt wird.

\subsection{Konkrete Modellierung des Meeresbodens}

Es gibt nur drei verschiedene Polygone, mit denen gleichmäßig
ohne Zuhilfenahme von Füllstücken parkettiert werden kann 
\ref{img:floor_comp_polygons} \footnote{Source}.

\begin{figure}
\includegraphics{img/dummy.png}
\caption{Diskretisierung des Spielfeldes mittels verschiedener Polygone}
\label{img:floor_comp_polygons}
\end{figure}

Der Meeresboden kann dann als Graph gesehen werden, bei denen Knoten 
als Zellen gesehen werden und geometrisch einem Polygon entsprechen. 
Angeordnet werden diese so, dass zwischen den Schwerpunkten zweier 
benachbarter Zellen exakt 1m Abstand ist. Dies hat den Grund, dass 
ein Roboter in jedem Zeitschritt in die Mitte der nächste Zelle wechseln kann.

Der Zusammenhang zwischen realer Umgebung und Modellierung kann in
Fig. \ref{img:floor_grap_polygons} 

\begin{figure}
\includegraphics{img/dummy.png}
\caption{Zusammenhang zwischen Meeresboden und Modellierung}
\label{img:floor_grap_polygons}
\end{figure}

Roboter werden so simuliert, dass die Bewegungen nur über die Kanten einer
Zelle möglich sind. Daher ist Anzahl der Ecken identisch mit den möglichen
Bewegungsrichtungen. Somit gilt: je mehr Ecken, desto mehr Freiheitsgrade.
Die Kanten geben an, von welchem Knoten zu welchen Nachbarn gewechselt werden
kann. Somit hat jeder Knoten auch so viele Kanten wie das gewählte Polygon
Ecken hat.

Da eine Simulation mit unterschiedlichen Polygonen unnötig komplex
wird, entscheidet es sich zwischen gleichschenkligem Dreieck, 
Quadrat und regelmäßigem Sechseck.

Der Arbeitsbereich eines Roboters ist kreisförmig, die Zellen, in denen er
sich befindet, jedoch ein Polygon. Daher gibt es in den Ecken der Zelle 
Bereiche, die nicht (unmittelbar) gesaugt werden. Berechnen lässt 
sich der Verlust über das Verhältnis von der Fläche des Innenkreises 
des Polygons zum Flächeninhalt des Polygons selbst. 

\begin{figure}
\includegraphics{img/dummy.png}
\caption{Innenkreis und Polygon}
\label{img:floor_quotient_circle_polygon}
\end{figure}

Die folgende Tabelle zeigt, wie viel Prozent einer Zelle nicht gesaugt werden,
je nachdem, welches Polygon gewählt wurde.

\begin{figure}
\includegraphics{img/dummy.png}
\caption{Fehlerrechnung}
\label{img:floor_error_circle_polygon}
\end{figure}

Die Auswahl fiel schließlich leicht, da die Simulation mit einem Quadrat als Polygon
folgende Vorteile bietet:

\begin{itemize}
\item Einfache Datenstruktur
\item Einfaches Berechnen der Nachbarn
\item Visualisierung entspricht pixeln, keine Umrechnung nötig
\item Weniger Verlust als Dreieck
\item Weniger Freiheitsgrade als Sechseck (weniger Auswahl bedeutet weniger Rechenaufwand)
\item Einfache Berechnung von Entfernungen zwischen zwei Zellen \ref{sec:manhattan}
\end{itemize}

Die Herleitung kann im Anhang \footnote{Anhang} nachvollzogen werden.

% ############################################
\clearpage
\section{Theory Crafting}

%#
\subsection{Manhattan-Metrik}
\label{sec:manhattan}

\subsubsection{Entfernung}


\subsubsection{Kreise}


%#
\subsection{Platzierung}

Keine Häufigkeitsverteilung gegeben

%#
\subsection{Missionsdauer}

goal runden
missiontime runden

%#
\subsection{Finden des Sammelpunktes}

%#
\subsection{Zeitbeschränkung}

% ############################################
\clearpage
\section{Implementierung}


% ############################################
\clearpage
\section{Diskussion}

\subsection{Performance}

\subsection{Fehlerrechnung}

\subsubsection{Abweichung vom Optimum}

\subsubsection{Fehler durch Modellierung}

\subsection{Lessons learned/Fehlentscheidungen}

\begin{itemize}
\item Zu viele Abstraktion, die sich später unnötig herausgetsellt hat (Geometry, Circle)
\end{itemize}

\subsection{Ausblick}


\end{document}
